% Kale Ewasiuk (kalekje@gmail.com)
% +REVDATE+
% Copyright (C) 2021 Kale Ewasiuk
%
% Permission is hereby granted, free of charge, to any person obtaining a copy
% of this software and associated documentation files (the "Software"), to deal
% in the Software without restriction, including without limitation the rights
% to use, copy, modify, merge, publish, distribute, sublicense, and/or sell
% copies of the Software, and to permit persons to whom the Software is
% furnished to do so, subject to the following conditions:
%
% The above copyright notice and this permission notice shall be included in
% all copies or substantial portions of the Software.
%
% THE SOFTWARE IS PROVIDED "AS IS", WITHOUT WARRANTY OF
% ANY KIND, EXPRESS OR IMPLIED, INCLUDING BUT NOT LIMITED
% TO THE WARRANTIES OF MERCHANTABILITY, FITNESS FOR A
% PARTICULAR PURPOSE AND NONINFRINGEMENT.  IN NO EVENT
% SHALL THE AUTHORS OR COPYRIGHT HOLDERS BE LIABLE FOR
% ANY CLAIM, DAMAGES OR OTHER LIABILITY, WHETHER IN AN
% ACTION OF CONTRACT, TORT OR OTHERWISE, ARISING FROM,
% OUT OF OR IN CONNECTION WITH THE SOFTWARE OR THE USE
% OR OTHER DEALINGS IN THE SOFTWARE.

%\documentclass[serif]{corpboreport}

\def\FormatDir{../class/}
%\documentclass[memo,serif]{\FormatDir corpboreportMulti}
%\documentclass[serif]{\FormatDir corpboreportMulti}
\documentclass{\FormatDir corpboreportMulti}


\setlength{\parindent}{0ex}
\newcommand{\llcmd}[1]{\leavevmode\llap{\texttt{\detokenize{#1}\ }}}
\newcommand{\cmd}[1]{\texttt{\detokenize{#1}}}
\newcommand{\qcmd}[1]{``\cmd{#1}''}
\newcommand{\llcmdlink}[2]{\href{#2}{\hspace{-15ex}}\hspace{15ex}\llcmd{#1}}%

\RequirePackage{url}
\RequirePackage{showexpl}
\lstset{explpreset={justification=\raggedright,pos=r,wide=true}}
\lstset{language=tex}

\setlength\ResultBoxRule{0.0pt}
\edef\LTXset{justification=\raggedright,pos=r,width=0.3\linewidth}

\BeforeBeginEnvironment{LTXexample}{\begin{addmargin}[-1.5cm]{2em}}
\AfterEndEnvironment{LTXexample}{\end{addmargin}}


%\secwiselabelnums

%\setlength{\subsecneedspace}{15em}

%\RenewDocumentEnvironment{lstlisting}{+b}{}{}  % todo something wrong with lstlisting... fix
%\RenewDocumentEnvironment{LTXexample}{+b}{}{}  % todo something wrong with lstlisting... fix



%! language = yaml
\begin{parseYAMLvars}
Title: |-
  a template for a rather
  corp{\smaller\smaller\normalfont orate}
  bore{\smaller\smaller\normalfont ing}
  \leavevmode\llap{{\smaller\smaller\normalfont re}}port

autHor:
 - John Smith
 - Jane Doe \url{kalekje@gmail.com}

subtitle: |-
		Some thought-provoking
		sub-title that inspires
		someone to read your
		boring report.

company: |-
  Some Department
  A University

logo: example-image-a

date: \today

rohead: ...
lohead: |-
  corpboreport
  documentation


pdf:
  keywords: [a,b,c]
  language: en-CA
  subject: a test document
  copyright: Kale Ewasiuk
  date: 2021-02-02
#  urllink: pass
\end{parseYAMLvars}
\writePDFmetadata


\begin{document}


%
%todo use /And command for authors? how to handle multiple? maybe use \{\} [] [] for multi authors?
%
%todo if at title or author (required meta data) not defined, should we throw an error telling the user that? Would be helpful
%
%todo adopt "string" cleaning for pdf string in MH report, clean the right header and set to pdf title, do same for author
%
%	TODO use \~ token for enlarging page with section command, probably a nice little easy thing to do
%
%	TODO pdf string stuff for yaml vars
%
%	TODO uni refs - make a lua table interface
	% https://tex.stackexchange.com/questions/4805/whats-the-correct-use-of-author-when-multiple-authors
	% https://tex.stackexchange.com/questions/466095/redefine-title-for-possibility-of-a-forced-line-breaking-in-a-predefined-place
%
%
%	\begin{LTXexample}
%		hello world
%
%		\begin{tabular}{ l !{--} l }
%			a & b \\
%		\end{tabular}
%	\end{LTXexample}

\maketitle


\IntentionallyBlankPage


\thetitle
\theauthor

\section{Quick Start}
\begin{itemize}
    \item Install MiKTeX (\url{https://miktex.org/download}) (do a single user installation)
    \item This report uses LuaLaTeX---ensure you use \texttt{lualatex} to compile and do not use the \texttt{pdftex}/\texttt{pdflatex} command to compile
    \item Strongly recommended to use \texttt{output-directory=./out} when compiling to push all generated files in a sub-folder
    \item For the CM-Bright (sans serif) font, you must go to MiKTeX console (hit Windows key and type miktex and it should pop up) and manually download hfbright and cm-super packages by clicking Packages (left menu) and searching, right-click, then install. Or you can run these commands: \texttt{mpm -\/-install=hfbright} and \texttt{mpm -\/-install=cm-super}
    \item you should also install markdown package: \texttt{mpm -\/-install=markdown}
    \item Install Perl for the makeglossaries package (\url{https://strawberryperl.com/}) and ensure perl is in the path (used for acronyms)
    \begin{itemize}
        \item Possibility: If you get an error in the \texttt{xindy.pl} script when trying makeglossaries, this means a certain folder cannot be found,
                and we must patch the code to get it working. I've found this issue occurs for admin type installations mostly. Use the error file and line and locate the issue. In the else part, the 'die' statement forces the crash.
                We must replace this with the following (ensure correct / direction is used):\\
                \texttt{\$cmd\_dir = "C:/Path/To/Where/tex2xindy.exe/is/located";}
        \item Common locations are:
        \begin{itemize}
            \item \url{C:/Users/Kale/AppData/Roaming/MiKTeX/2.9/miktex/bin/x64}
            \item \url{C:/Program Files/MiKTeX/miktex/bin/x64/internal}
        \end{itemize}
    \end{itemize}
\end{itemize}


Important note for beginners: When you install MiKTeX, you get a LaTeX installation and package manager.
This operates independently of what tool you use to write LaTeX in (TeXStudio, TeXWorks, PyCharn, VScode, notepad).
To compile a LaTeX document, you simle need to go to the command line and run a command that more or less says "compile this file".
The tool that you use to write you code in just helps execute these commands, captures the output from the compilation process,
and helps point to any errors.

This package uses lualatex instead the default pdftex command.
You should be able to modify TeXStudio or VScode (example below) or whatever you use to make this the default. I also like to push the .pdf, and other auxiliary files to an 'out' folder.

Recommended compile command:\\ \cmd{lualatex -file-line-error -interaction=nonstopmode -synctex=1 -output-format=pdf -output-directory=out main.tex}

Biblography: \cmd{biber main --output-directory ./out}

Glossary/Acronym: \cmd{makeglossaries -d ./out main}

PDF viewer: I like Sumatra as you can leave a PDF open in it and still re-compile (Adobe hogs the file).\url{https://www.sumatrapdfreader.org/free-pdf-reader}

\pagebreak

\subsection{Compiling with VSCode}

Open VSCode and the \cmd{settings.json} file. Add the following code. If the outer-most brackets aren't there, copy this verbatim.
If there are brackets, skip the outermost brackets and place the code inside.
\enlargethispage*{3em}
\begin{lstlisting}

{
    "latex-workshop.latex.recipes": [
        {
            "name": "compile lualatex",
            "tools": [
                "mylualatex",
            ]
        },
        {
            "name": "compile bib gloss",
            "tools": [
                "mybiber", "myglossary",
            ]
        },

    ],
    "latex-workshop.latex.tools": [
        {
            "name": "mylualatex",
            "command": "lualatex",
            "args": [
                "-synctex=1",
                "-interaction=nonstopmode",
                "-file-line-error",
                "-output-format=-pdf",
                "-output-directory=%DIR%/out",
                "%DOCFILE%"
            ]
        },
        {
            "name": "mybiber",
            "command": "biber",
            "args": [
                "main",
                "--output-directory", "out",
            ]
        },
        {
            "name": "myglossary",
            "command": "makeglossaries",
            "args": [
                "-d", "out",
                "main"
            ]
        },
    ]

}
\end{lstlisting}

\section{Introduction}

This class is based off the \cmd{scrartcl} class in the KOMA-Script family.
Don't let the name fool you---although plain and unassuming with a (dare I say) Microsoft Word-like appearance,
this template has numerous bells and whistles that make your \LaTeX ing easier.
It could easily lend itself well to school assignments or lab reports as well.



\section!{Class Options}
\llcmd{memo}use a memo format instead.\\
\llcmd{compact}use a compact format  (sections not on a new page, and different title page.)\\
\llcmd{serif}change the font to serif (kpfonts light instead of CM-Bright).


\section!{Switches}

\begin{lstlisting}
%%% Draft toggles
\DarkModePDF %< if you want a dark theme
\DraftMark %< puts a draft water mark
\PreLimMark %< auto set

%%% some formatting toggles
\togglefalse{SecOnNewPage}  % insert new page before each section?
\secwiselabelnums % if you want figures and table numbers to be based on section, ie 1.1, 1.2, etc.

\providelength{\tpsignheight}{7cm}  % adjust height of tp area

\setlength{\tocsecskip}{0.7em}  % if you want to spread or compress ToC to fit page better
\setlength{\pullToCcloser}{-12pt} % pulls ToC closer to section heading for more room
\setlength{\moreToCbottom}{2em} % enlarges ToC page to help cram stuff

\toggletrue{showListOfFigsTabs} %< show figs/tables contents
\togglefalse{samepgListOfFigsTabs} %< on the same page?
\end{lstlisting}


\section!{Some Basic Settings}
The title of this document is redefined, and uses the KOMA commands
\cmd{\title{}}
\cmd{\subtitle{}}
\cmd{\author{}}
\cmd{\date{}}
\cmd{\company{}} (equivalent to publishers)
\cmd{\logo{}} (added, path to graphics) to set some title page parameters

Since this document class uses KOMA Script, change the header with
\cmd{\lohead{}, \rohead{}}


\section{Sections}\label{s.sett}

This class by default will insert a page break on a new section.
Using the etoolbox package, you can disable this behaviour.
if you want to invert the behaviour for a particular section, use !

For paragraphs, if you would like the text to appear on a new line (rather than run in), use !.

\begin{lstlisting}
use [short title for ToC]{Section title}
		* will suppress number and ToC listing
\section{}
\subsection
\subsubsection
\paragraph
Custom
\section! will not necessarily put the section on a new page. By default, sections go on a new page
\paragraph!   Will push the following text on a new line. Without "!", the text begins on the same line as the heading
On any of the section/sub commands, using a + will add a letter after it.
\end{lstlisting}

The sections are redefined so that \texttt{+} will append a letter. \cmd{\reset[sub]sectionletter}

\begin{LTXexample}
\subsection{sub}
\subsection+{sub plus}
  \subsubsection{subsub}
  \subsubsection+{subsub plus}
\subsection+{sub plus}
\resetsubsectionletter
\subsection+{reset, then sub plus}
\end{LTXexample}


Add a manual page break in ToC \cmd{\addtocontents{toc}{\protect\newpage}}




\section{References and Citations}

\subsection{References}

I use the facilities of the cleveref package, which automatically type Figure Table etc
I like using s. f. t. etc. to label my things, it's easier to type than the typically recommended sec: fig: tab:
Examples:
\cmd{\cref{f.m602fTOV}}
\cmd{\label{s.steadystate}}

\subsection{Citation}

The .bib files contain the reference information. Call the key with \cmd{\cite{key}}.

I have a command that prints the title then adds the reference (in italic): \cmd{\citeT{key}}

I like JabRef to help manage my .bib files \url{https://www.jabref.org/}


\section{Float commands}

This package offers float commands for inserting tables and figures

\begin{lstlisting}
\InsertTable[htbp]%
{you can \input here or type \begin{tabular}...}%
{Caption goes here: egManitoba interface transfers.}%
{\label{t.trans}}% label goes here
[optional]%Optional table footnotes here

\InsertFigure[htbp]%
{\includegraphics[options]{path/to/figure.pdf}}%
{A long run on caption that is needlessly long}%
{\label{f.1}}%

NOTE: you can put a *  directly after
InsertTable or InsertFigure,
and it will make it a wide table/fig
that is flush left, no star will indent the float a bit
\end{lstlisting}


Some helpful commands.

\begin{lstlisting}

\RotPDF*+{\include...}
	* star will restore the pdf back to normal orientation
	+ will add a pagebreak before

\RotFloatPage{\Insert [H]...}
   If you want a rotated float (figure or table) on its own page, use this with the [H] float placement

\FloatNextPage{\Insert [H]...}
  If you have a big float and you know it should be on its own page right after

\BoxSameSizeImg{O{t} O{t} m m}
  make a box the same size of an image
\end{lstlisting}



\section{Tabular matter}

This class uses the author's \cmd{lutabulartools} package, and the author highly suggests we follow
the \cmd{booktabs} way of making tables (no vertical lines).


\subsection{References}

Tabular 101:\url{https://en.wikibooks.org/wiki/LaTeX/Tables}\\
How to make nice tables: \url{https://inf.ethz.ch/personal/markusp/teaching/guides/guide-tables.pdf}

\subsection{Columns}

	\begin{tabularx}{\linewidth}{l Z}\toprule
		Column & Use \\\midrule
		L,N,R\{X.Y\} & \cmd{siunitx} number column (where X.Y is number format)\\
		X,Z,Y & tabularx column, X=justified, Z=ragged (usually preferred),\nl or Y=centered\\
		P,M,B \{wid\} & ragged instead of justified equiv to p,m,b\\
	  V, T \{wid\} & horizontally centered plus vertically centered (V) / top aligned (T) paragraph cell\\
		\cmd{~}\phantom{1} & inject default tablcolsep, equiv to \cmd{@{\hspace{\tabcolsep}}}\\
		\\
  	\MC[2]{The defaults}\\
		S & see siunitx doc \\
		l,c,r & Left, center, right, fits to width\\
	  p,m,b \{wid\} & A top, middle, bottom aligned paragraph cell, allows \cmd{\newline and \nl}\\
		\bottomrule
	\end{tabularx}


In a paragraph cell (ie p, P, X, Z, Y, you can specify line-breaks with \cmd{\newline} or a short version \cmd{\nl} (defined in this class))

Rules (lines)
\begin{lstlisting}
\toprule
\midrule
\bottomrule		\cmidrule(){}
\gmidrule		custom light gray mid rule
\end{lstlisting}

Footnotes
\begin{lstlisting}
tnote{ltr}:
tfnote{ltr}:

\reseturef or \resetatnotes - reset automatic lettering
atnote{key}: automatic table note
atfnote{key}: automatic table footnote
\end{lstlisting}


Note: if using a float,
After end{tabular}\%\\
Make sure a \% and no new lines after, this improves

\subsection{Hacks}

{\toglnums 1234567890 abcdefghi}

\subsubsection{Spacing}
Use the following commands outside the tabular environment.
For different vertical spacing, use \cmd{\renewcommand{\arraystretch}{1.2}}.\\
For different spacing, \cmd{\setlength\tabcolsep{2ex}}

\subsubsection{Columns}

If you have a multicolumn (say by using \cmd{\MC}), and the multicolumn is wider than the combined width of your
columns underneath and they were not fixed-width columns (p,P,m,M,b,B,T,V,X,Z,Y), if you want them to be the same width,
just use one of the aforementioned fixed-width columns. You might need to tweak a bit though.

Out of convenience, I erase the padding on the ends of tabular, because I like the way it looks.
To bring it back with the \cmd{~} column like  \cmd{~lllr~}. Here's an example:

\begin{LTXexample}
\begin{tabular}{~ll~}\toprule
hello & world \\
\bottomrule
\end{tabular}

\begin{tabular}{ll}\toprule
hello & world \\
\bottomrule
\end{tabular}
\end{LTXexample}


Use \cmd{@{}} to remove space between columns, or \cmd{@{'code'}} to insert code between column
\cmd{!{code}} adds code keeps the space though.\\
Use \cmd{>{'code'} <{'code'}} to sandwich a cell with code of your choice

Note: with the siunitx columns N, L, R type columns, need to surround text with \cmd{{}} (and text that may come after a number) to get alignment correct

	


\subsection{Common Errors}
Common Errors
	Extra \textbackslash cr	You have too many \& or forgot to put a \textbackslash\textbackslash to end your row
	siunitx invalid input	You probably forgot to wrap text wit \{\}  (note that you should not wrap multicolumn with \{\})
	tex capacity exeeded	put \{\} after midrule ?
	\textbackslash MC on a p\{\} column?

Misplaced noalign -if you use \cmd{\midrule} etc before the \cmd{\\} you will get this error

Misplaced omit or span if you use a multicolumn or non-number column in an siunitx column and
forget to wrap with \{\}

\section{List commands}
\url{https://texblog.org/2008/10/16/lists-enumerate-itemize-description-and-how-to-change-them/}

\begin{lstlisting}
\ContParaAfterList	If you're showing a list (itemize) and want the list to be in the same paragraph as the next lines, use this command to fix the spacing
	Otherwise, it will assume a new pagagraph follows the list and space will be larger
Lists	\ begin{itemize/enumerate}[moresep] will give more room between items
	[twocol] will put them over two columns

\end{lstlisting}

	Added an autoamtically punctuate list. This is particularly useful if you have a large list and aren't entirely sure of the order

\begin{itemize}[autopunc]
	\item one
	\item two
	\item three
	\item four
\end{itemize}

	\section{Utilities}
in addition to etoolbox,we have
\cmd{\invtoggle{}}, \cmd{\gettogglestate}, \cmd{\ifdefOR{d1}{d2}{t}{f}},
\cmd{\ifstringeqx{}{}{}[]}, \cmd{\DoIfnotEmpty{}{}[]},
\cmd{\DoWithoutPrinting{}},
\cmd{\DontDo{}},

Some other stuff:

\cmd{\makealph{num}}
\cmd{\makeAlph{num}}

\begin{lstlisting}
\hl{highlight text}
\todo{todo note in margin}
\todoL{todo note on a new line}
\end{lstlisting}


\section{Hacks}

With KOMA-Script, we can make margin wider or smaller:\\ \cmd{\begin{addmargin}[left indentation]{indentation} ... \end{addmargin}}

\cmd{\phantom{xx}} makes an invisible character of same size

\cmd{\llap{}, \rlap{}, \clap{}}	allows text to have no width (ie overlaps), left, right or center aligned
\cmd{\llap{}} useful for left hanging labels (section, fig, tables), but these in general are good for
hacking tables to create an item of zero width which will not adjust column width.
If you want to use these as the first thing in a paragraph, use \cmd{\Llap, \Rlap, and \Clap},
otherwise, you may get some unwanted space.

Horizontal spacing: I use \cmd{\,} to produce a thin space: good for initials, or units like \cmd{K.\,Ewasiuk, 177\,lbs}

\cmd{\enlargethispage} \url{https://latexref.xyz/_005cenlargethispage.html}

\url{https://tex.stackexchange.com/questions/74353/what-commands-are-there-for-horizontal-spacing}


If you need to add a forced page break, we have some options. These should be done at the very end of the report writing process though.

\url{https://tex.stackexchange.com/questions/9852/what-is-the-difference-between-page-break-and-new-page/9855#9855}
pagebreak -- create a new page and fill the page\\
newpage  -- perform a new page and don't let the paragraphs spread\\
clearpage -- perform new page but ensure all floats are inputted\\



\section{Typography}

\begin{lstlisting}



Typing units like kV, MW
	Use a backslash in front. I define units using the siunitx package
	\kV  \MW
	This produces a smaller space and keeps the units together
\end{lstlisting}


\subsubsection{Know your dashes}
\begin{LTXexample}
- is a hyphen: old-timers\\
-- is an en-dash and used for ranges like: 0--60\\
--- is an em-dash: Kale---the cool guy---said that
       em-dashes are cool!
\end{LTXexample}



\subsubsection{Know how spaces, new lines, and comments work in LaTeX}
\begin{LTXexample}
This will all
be one sentence.

\ \ \ \ New paragraph from blank line.
Supercali%<<see percent/comment
fragilstitic%<<<
expilaidocuious  % << no space

\ \ \ \ It is useful to break ideas
into separate lines.
And makes it easy to
move lines around
if you change you mind, or
parse out lists like
cat,
dog, and
fish.
\end{LTXexample}




\section{Packages}

Here are some packages which I think the documentation is worth reading.

\llcmd{V links V}


\llcmdlink{xparse}{https://muug.ca/mirror/ctan/macros/latex/contrib/l3packages/xparse.pdf}The smart way to define commands and environments.\\
\llcmdlink{etoolbox}{http://mirrors.ctan.org/macros/latex/contrib/etoolbox/etoolbox.pdf}Useful ``scripting'' tools like toggles, if-else, and hook into commands and environments.\\

\llcmdlink{booktabs}{http://mirrors.ctan.org/macros/latex/contrib/booktabs/booktabs.pdf}Better rules(lines) for tables\\
\llcmdlink{lutabulartools}{https://muug.ca/mirror/ctan/macros/luatex/generic/lutabulartools/lutabulartools.pdf}Provides \cmd{\MC} and enhances booktabs.\\You may want to see \hspace{10ex}
\llcmdlink{makecell}{http://mirrors.ctan.org/macros/latex/contrib/makecell/makecell.pdf} and \hspace{10ex}
\llcmdlink{multirow}{https://ctan.mirror.globo.tech/macros/latex/contrib/multirow/multirow.pdf} as well.\\
\llcmdlink{tabularx}{http://mirrors.ibiblio.org/CTAN/macros/latex/required/tools/tabularx.pdf}The tabularx environment.\\
\llcmdlink{longtable}{https://ctan.mirror.rafal.ca/macros/latex/required/tools/longtable.pdf}table that can extend past one page\\
\llcmdlink{ltxtable}{https://ctan.mirror.globo.tech/macros/latex/contrib/carlisle/ltxtable.pdf}Long tabularx

\llcmdlink{siunitx}{https://texdoc.org/serve/siunitx/0}Units and numbers in text and tables.

\llcmdlink{enumitem}{http://mirrors.ctan.org/macros/latex/contrib/enumitem/enumitem.pdf}Fancy lists and key-val options.\\
\llcmdlink{autopuncitems}{https://mirror.csclub.uwaterloo.ca/CTAN/macros/luatex/latex/autopuncitems/autopuncitems.pdf}Automatically punctuate lists

\llcmdlink{floatrow}{http://mirrors.ctan.org/macros/latex/contrib/floatrow/floatrow.pdf}Used to help with the InsertTable/Figure.\\
\llcmdlink{caption}{https://mirror.csclub.uwaterloo.ca/CTAN/macros/latex/contrib/caption/caption-eng.pdf}Tweak captions.\\
\llcmdlink{graphicx}{https://ctan.mirror.rafal.ca/macros/latex/required/graphics/grfguide.pdf}enhanced includegraphics\\

\llcmdlink{cleveref}{http://tug.ctan.org/tex-archive/macros/latex/contrib/cleveref/cleveref.pdf}Allows cref and coustomization of captions\\

\llcmdlink{relsize}{https://mirror.csclub.uwaterloo.ca/CTAN/macros/latex/contrib/relsize/relsize-doc.pdf}Provides relative sizing commands like \cmd{\smaller}.\\

\llcmdlink{KOMA-Script}{https://texdoc.org/serve/KOMA-Script/0}You shouldn't have to mess around with this,
but if you wanna tweak the class go for it.

\llcmdlink{yamlvars}{https://ctan.mirror.globo.tech/macros/luatex/latex/yamlvars/YAMLvars.pdf}Make definitions with yaml.
Used for report variables


\llcmdlink{LuaTeX}{https://www.pragma-ade.com/general/manuals/luatex.pdf}The reference manual.


\section{Hydro Stuff}

Note: the settings of \cref{s.sett} do not apply to Hydro reports.

\subsection{Report Variables}
This package can use YAMLvars.

\subsection{Interconnection Study Macros}

\begin{lstlisting}
Hydro Report Macros	R:\TariffStudies\MH-Report-Template\_boilerplate\HydroMacros.sty

\Study	Produces ies, ifs, sis, fs (or group in front) and enters glossary item
\aStudy	an ies, a fs, etc. does this correctly
\Agreement	Same as above but with agreement
\GenStation	Types out the generating station
\Request	Puts IR or TSR
\Firm	Puts NRIS if it's a generator study, DNR if it's a tariff study
\SecS	Puts NRIS if it's a generator study, DNR if it's a tariff study
\ifIFS{x}[y]	Types x if study type is ifs, otherwise types y (optional)
\ifGroupStudy
\ifTariffStudy
\PSSE	\Mhtrans (mh transmission system)
\Customer	IC for oait, EC for oatt
\IntUpgrades	isus and toids





When typing acronyms	Excel spreadsheet is used to manage acronyms
	Best to use \gls{oait}
	For example. This will ensure Open Access Interconnection Tariff is only typed the first time

\end{lstlisting}

\PrintEndOfDocument

\end{document}