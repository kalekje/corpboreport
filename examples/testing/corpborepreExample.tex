% the sample slide is created with 16:9 aspect ratio

\def\BeamerOptions{%
%	handout,
%	draft,
%	aspectratio=43,
%	notes,
%	hide notes, % Only slide
%	show only notes, % Only notes
%	show notes on second screen=right, % Both
}


\def\FormatDir{./../../class/}
\documentclass{\FormatDir corpborepresMulti}

%\documentclass{corpborepres}


%%% Switches (todo make these class settings)
%\sectionalToC
%\allowframebreaksall
%\DarkTheme
%\subtitleINline %\subtitleNEWline


%https://tex.stackexchange.com/questions/160767/how-to-customize-block-in-beamer
% https://tex.stackexchange.com/questions/372586/beamer-reduce-top-margin-which-is-possibly-left-for-title
%\setlength{\beamer@headheight}{2in}
%\setbeamersize{text margin left=30mm,text margin right=30mm}
%https://tex.stackexchange.com/questions/270409/adjust-vertical-height-of-beamer-box-title-bar
%\newcommand{\addheight}{\parbox{0pt}{\rule{0pt}{5cm}}}
%\setbeamerfont*{block title}{family=\sffamily,series=\bfseries\addheight,size=\Huge}
%\setbeamerfont*{frame title}{family=\sffamily,series=\bfseries\addheight,size=\Huge}

%https://tex.stackexchange.com/questions/368710/beamer-how-to-get-height-for-frame-content
% https://tex.stackexchange.com/questions/278429/is-there-a-simple-command-for-the-available-height-in-a-beamer-slide/278434#278434
\NewDocumentCommand{\FitToHeight}{ m }{
\makeatletter
  \global\beamer@shrinktrue
    \gdef\beamer@shrinkframebox{
        \setbox\beamer@framebox=\vbox to\beamer@frametextheight{
			\gdef\HeightMax{height=\beamer@frametextheight}%
			#1%
        }%
    }%
\makeatother
}

% the background and logo are in the images directory
\graphicspath{{images/}}

% information for the title page
\author{Kale Ewasiuk}
\title{A Corporate Boring Presentation\\ Longassish}
\subtitle{... that makes little use of color, and is not very fun!}
\company{Grid Infrastructure Planning Dept.}
\date{\today}
\logo*{example-image-a}
%\logopos{} % todo make logo position option, referenced from bottom

\begin{document}
	\begin{frame}[plain]
		TODO: alert transitions aren't working.
		Incremental alert with graphics and caption not working...
		TODO: make symbols for caution (like fourier), and example bolt, brain, or lightbulb? hmmm maybe a key for takeaway
		% https://thenounproject.com/icon/key-655/
		% caution, key, calculator ??
%		https://www.google.com/amp/s/nhigham.com/2013/01/18/top-5-beamer-tips/amp/
	\end{frame}

	\begin{frame}{Test tikzmark}
		a\tikzmark{a}
		a\tikzmarknode[shift={(1,1)}]{a}{}%todo should tinker with the settings for position...
%		a\tikzmarknode[inner sep=0,shift={(1,1)}]{a}{}%todo should tinker with the settings for position...
%		a\tikzmarknode[inner sep=0,scope={shift={(1,1)}}]{a}{}%todo should tinker with the settings for position...
%		a\tikzmarknode[inner sep=0, pos={(0,1.2ex)}, yshift=1.2ex]{a}{}%todo should tinker with the settings for position...
%		a\tikzmarknode[inner sep=0, xy={(0,1.2ex)}]{a}{}%todo should tinker with the settings for position...
%		a\tikzmarknode[inner sep=0, pos={(0,1.2ex)}]{a}{}%todo should tinker with the settings for position...

		all the way to \tikzmark{b}b
		all the way to \tikzmarknode[inner sep=0, at={(0,-1.2ex)}]{b}{}b  % https://tex.stackexchange.com/questions/396288/positioning-node-in-tikz-and-some-more-questions

		\tikzmarknode{N}{\Huge The qyick brown rfo}

		\tikz[OL] \draw (a) -- (b);  % works with tikzmarknode !
		\tikz[OL] \draw (b) -- (N.north east);  % works with tikzmarknode !
%		\tikz[OL] \draw (pic cs:a) -- (pic cs:b);
%		\tikz[OL,cs=pic] \draw (a) -- (b);
%		\tikz[OL,shift={pic}] \draw (a) -- (b);
%		\tikz[OL,transform={pic}] \draw (a) -- (b);
	\end{frame}

	\begin{frame}[plain]
		\titlepage
	\end{frame}

	\begin{frame}{Outline}
		\tableofcontents
	\end{frame}

	\section{Basics}

	\subtitleINline
%	\begin{frame}{Blocks}{This is a subtitle in line}
%
%		\begin{block}{Standard Block}
%			This is a standard block. %\note{holy shit}
%		\end{block}
%
%		\begin{exampleblock}{\Llap{ex.\ }Example Block}
%			This is an example block.
%		\end{exampleblock}
%
%		\begin{alertblock}{\Llap{\danger\ }Alert Block}
%			This is an alert block.
%		\end{alertblock}
%	\end{frame}
	\subtitleNEWline
	\begin{frame}{Testin paragraphs}
		Hello world. adsa dsa ds as .\\New Line.

		How do you do.

		todaty.
	\end{frame}

\begin{frame}
		\includegraphics[width=\linewidth]{example-image-a}
		\freecaption{Yolo\footnote[frame]{A test footnote in the second column}}
\end{frame}


	\begin{frame}[plain]
		\begin{tikzpicture}[overlay,remember picture]
			\node[nopad,anchor=north east]  (a) at (current page.30){\includegraphics[width=3cm]{example-image-a}};
			\node[nopad,anchor=south] (b) at (current page.south) {\includegraphics[width=3cm]{C:/Users/Kale/Projects/pltx/testing/dum-crop.pdf}};
			\draw (a.south) -- (b.north);
			\draw[thick] (current page.south west) rectangle (current page.north east);
			\node[nopad,draw,anchor=west] at ($(current page.west)+(0, -3)$) {\Large wtf mate};
		\end{tikzpicture}
	\end{frame}

\begin{frame}
\twoCol{\begin{itemize}
            \item 1
            \item 2
            \item 3
         \end{itemize}}
 	\begColFF
		hello\hfill world!
	\nextCol
	\begin{itemize}
            \item 1
            \item 2
            \item 3
         \end{itemize}
	\endCol
\end{frame}

%\begin{frame}  % todo not working
%\FitToHeight{\includegraphics[rotate=90,\HeightMax]{example-image-a}}
%\end{frame}

\begin{frame}[allowframebreaks]{Buncha Columns}
	\arabic{framenumber}
	\begColFF
			\begColFF
				hello world
					\begin{itemize}\moresep
						\begin{AutoPuncItems}
							\item Hello
							\item World
							\item this sucks
						\end{AutoPuncItems}
			\end{itemize}
			\nextCol
				splitting a col
			\endCol
	\nextCol
			\begColFF
				into 4
			\nextCol
				fucking pieces!
			\endCol
	\endCol

	\framebreak

	Holla! 	\arabic{framenumber}

\end{frame}
	\url{https://tex.stackexchange.com/questions/61873/differences-and-best-practices-onslide-vs-uncover-onslide-vs-visible}
	\begin{frame}{\ }
		\begin{tabular}{lcl}\toprule
			\tmrk[t]{2a}a & \tmrk[t]{1a}b2 & c\tmrk[b]{2b}\\\midrule
			\tmrk[m]{3}1 & 2222 & 3\\\midrule
			x & y & z\tmrk[b]{1b}\\\bottomrule
		\end{tabular}
%		\uncover<2->{\drawmark{a1}}
		\onslide<2->{\drawOverlayBoxTLBR{1a}{1b}}
		\onslide<3->{\drawOverlayBoxTLBR{2a}{2b}}
		\tikz[OL] \node [left = 0.0cm of tm-3]  (E) {instructions};
	\end{frame}

\begin{frame}{\ }
	\def\imgnumX{6}
	\begColNW
		\begin{itemize}
			\item<+-|alert@+> I am about to...
			\item<+-> reveal something so cool...
			\item<+-> it will knock your socks off.
			\item<+-> {\Huge It's a graphic!!!}
			\item<alert@\fpeval{\imgnumX-1}-\fpeval{\imgnumX}|\fpeval{\imgnumX-1}->  Here's the starting point:\tmrk*[m]{p1}
		\end{itemize}
		*(the author is not liable for seizures due to excitement)
	\nextCol
%		\drawHelpingGrid%
		\tmrk*(0.75,1.2){p2}\includegraphics<\imgnumX->{C:/Users/Kale/Projects/pltx/testing/dum-crop.pdf}
%		\tmrk(0,0){p}\includegraphics{C:/Users/Kale/Projects/pltx/testing/dum-crop.pdf}
		\freecaption<|\imgnumX->{Hello world.}
	\endCol
	\tikz[OL,nopad,visible on=<\fpeval{\imgnumX+2}->] \draw[ultra thick,pink,->] (tm-p1) to[bend left] (tm-p2);
%	\tikz[OL] \draw[blue] (tmrkr-p2) circle (2pt);
\end{frame}

	\section{One}

	\subtitleINline
	\begin{frame}{Slide Full of\\Lists (no break)}{subtitle}
	Saarland is a German state in the dynamic border triangle of Germany, France and Luxembourg.
	\cref{test}
		\begin{itemize}
			\item Saarland in figures
				\begin{description}
					\item[Federal state since] 1st January 1957
					\item[Area]: 2,569.69 $\text{km}^2$
					\item[Highest mountain]: 695m (Dolberg in Hunsrück)
					\item[Population]: 995,597 (31st December 2015)
					\item[Unemployment rate]: 6.5\% (June 2017)
				\end{description}

				\begin{itemize}
					\item \textbf{Federal state since}: 1st January 1957
					\item \textbf{Area}: 2,569.69 $\text{km}^2$
					\item \textbf{Highest mountain}: 695m (Dolberg in Hunsrück)
					\item \textbf{Population}: 995,597 (31st December 2015)
					\item \textbf{Unemployment rate}: 6.5\% (June 2017)
				\end{itemize}
			\item History
				\begin{itemize}
					\item \textbf{1960--1987}: France forms a Saar province as part of its reunification policies
					\item \textbf{13th January 1935}: Referendum reinstates Saarland into the Third Reich
					\item \textbf{1947}: Saarland is annexed to France in economic terms
					\item \textbf{1957}: Saarland becomes the 10th Federal state of the Federal republic of Germany
					\item \textbf{1957}: Saarland becomes the 10th Federal state of the Federal republic of Germany
					\item \textbf{1957}: Saarland becomes the 10th Federal state of the Federal republic of Germany
					\item \textbf{1957}: Saarland becomes the 10th Federal state of the Federal republic of Germany
				\end{itemize}
			\item Credits: \url{https://www.saarland.de/}
		\end{itemize}
	\end{frame}
	\subtitleNEWline


	\begin{frame}[allowframebreaks]{Slide Full of Lists}{Oops another subt}

	\arabic{framenumber}

	Saarland is a German state in the dynamic border triangle of Germany, France and Luxembourg.
	\cref{test}
		\begin{itemize}
			\item Saarland in figures
				\begin{itemize}
					\item \textbf{Federal state since}: 1st January 1957
					\item \textbf{Area}: 2,569.69 $\text{km}^2$
					\item \textbf{Highest mountain}: 695m (Dolberg in Hunsrück)
					\item \textbf{Population}: 995,597 (31st December 2015)
					\item \textbf{Unemployment rate}: 6.5\% (June 2017)
				\end{itemize}
			\item History
				\begin{itemize}
					\item \textbf{1960--1987}: France forms a Saar province as part of its reunification policies
					\item \textbf{13th January 1935}: Referendum reinstates Saarland into the Third Reich
					\item \textbf{1947}: Saarland is annexed to France in economic terms
					\item \textbf{1957}: Saarland becomes the 10th Federal state of the Federal republic of Germany
					\item \textbf{1957}: Saarland becomes the 10th Federal state of the Federal republic of Germany
					\item \textbf{1957}: Saarland becomes the 10th Federal state of the Federal republic of Germany
					\item \textbf{1957}: Saarland becomes the 10th Federal state of the Federal republic of Germany
					\item \textbf{1957}: Saarland becomes the 10th Federal state of the Federal republic of Germany
					\item \textbf{1957}: Saarland becomes the 10th Federal state of the Federal republic of Germany
					\item \textbf{1957}: Saarland becomes the 10th Federal state of the Federal republic of Germany
					\item \textbf{1957}: Saarland becomes the 10th Federal state of the Federal republic of Germany
					\item \textbf{1957}: Saarland becomes the 10th Federal state of the Federal republic of Germany
					\item \textbf{1957}: Saarland becomes the 10th Federal state of the Federal republic of Germany
					\item \textbf{1957}: Saarland becomes the 10th Federal state of the Federal republic of Germany
					\item \textbf{1957}: Saarland becomes the 10th Federal state of the Federal republic of Germany
					\item \textbf{1957}: Saarland becomes the 10th Federal state of the Federal republic of Germany
					\item \textbf{1957}: Saarland becomes the 10th Federal state of the Federal republic of Germany
					\item \textbf{1957}: Saarland becomes the 10th Federal state of the Federal republic of Germany
					\item \textbf{1957}: Saarland becomes the 10th Federal state of the Federal republic of Germany
				\end{itemize}
			\item Credits: \url{https://www.saarland.de/}
		\end{itemize}
			\arabic{framenumber}
			\insertframenumber
			\makeatletter
				% https://tex.stackexchange.com/questions/275044/how-do-i-insert-the-total-continuation-count-in-the-allowframbreaks-frame-title
%				\beamer@pageofframe
			\makeatother

	\end{frame}


	\section{Transitions}

\begin{frame}{Test}
		\label{test}
		\addfootnotespace
		\begColFF
			\vspace*{-1ex}
			Hmmm\footnote[frame]{A test footnote in the second column}
			\begin{itemize}
%			\begin{AutoPuncItems}
				\item <1-> Hello
				\item \alert<2-> {World}
				\item <3-> this sucks
			    \item <.-|alert@.> Foo
    			\item <+-|alert@+> Bar
%			\end{AutoPuncItems}
			\end{itemize}

		\begin{tabular}{l l l }\toprule
		Header & Num & Comment \\\midrule
		Apple & 3 & PinkLady\tnote{oi} \\
		\bottomrule
		\end{tabular}
		\freecaption{\tfnote{oi}The quick brown fox.}

		new paragraph. Yadaday. Yaday a new parag.
		\vspace*{1ex}

		\nextCol
			\onslide<3->{
				\includegraphics[width=\linewidth]{example-image-a}
				\freecaption{This is an image. \footnote[frame]{\url{www.google.ca}}}
			}
		\endCol

	\end{frame}


	\section{Code Blocks and Math}

	\begin{frame}{Math}
		Mathematics is the queen of sciences and arithmetic is the queen of mathematics.
		\begin{align*}
			\Pr(Y \geq 120) &= \Pr\left(Y-n\mu \geq 120-n\mu \right)\\
			&= \Pr\left( \frac{Y-n\mu }{\sqrt{n}\sigma} \geq \frac{120-n\mu }{\sqrt{n}\sigma} \right)\\
			&=\Pr\left( Z \geq \frac{120-n\mu }{\sqrt{n}\sigma} \right)\\
			&=\Pr\left( Z \geq \frac{120-100 \cdot 1 }{10 \cdot 1} \right)\\
			&=\Pr\left( Z \geq 2\right)
		\end{align*}
	\end{frame}



\begin{frame}[allowframebreaks,fragile]{Some Code}  %%% note the fragile!!!!
	\url{https://tex.stackexchange.com/questions/28229/extend-a-language-with-additional-keywords}
	\begin{lstlisting}[language=Python, name=code]
import numpy as np

def incmatrix(genl1,genl2):
	what = "do some python stuff"
    m = len(genl1)
    n = len(genl2)
    M = None #to become the incidence matrix
    VT = np.zeros((n*m,1), int)  #dummy variable

    #compute the bitwise xor matrix
    M1 = bitxormatrix(genl1)
    M2 = np.triu(bitxormatrix(genl2),1)

    for i in range(m-1):
        for j in range(i+1, m):
            [r,c] = np.where(M2 == M1[i,j])
            for k in range(len(r)):
                VT[(i)*n + r[k]] = 1;
    			...
                if M is None: M = np.copy(VT)
                else: M = np.concatenate((M, VT), 1)
                VT = np.zeros((n*m,1), int)
    return M
\end{lstlisting}

	\tikz[OL] \draw[->] (pic cs:line-code-5-first) -- (pic cs:line-code-7-first);
	\tikz[OL] \draw[->] (0,0) -- (pic cs:line-code-21-first);

\end{frame}

	\begin{frame}{Two Columns}
		We can also add two columns in the slides.\footnote[frame]{A random note.} bah ;adsa dsadsa dsadas.
		\begColFF
				This is the first column. In this column, we can also add a block for instance.
				\vspace{1em}
				\begin{block}{Block}
					I am a block in a column.
				\end{block}
			\nextCol
				\begin{itemize}
					\item In this column,
					\item we just add the
					\item bullet points.
				\end{itemize}
			\endCol
	\end{frame}

	\begin{frame}{Acknowledgements}
		This theme is inspired by Flip theme (creator: Flip Tanedo). The official beamer user guide was also very handy during the development.
	\end{frame}

	\begin{frame}[allowframebreaks]{References}
%		\bibliographystyle{plain}
%		\bibliography{}
	\end{frame}


\end{document}
